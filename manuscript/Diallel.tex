% -/\/\/\/\/\/\/\/\/\/\/\/\/\/\/\/\/\/\/\/\/\/\/\/\/\/\/\/\/\/\/\/\/\/\/\/\/\/\/\/\/\/\/\/\/\/\/\/\/\/\/\/\/\/\/\/\/\/\/\/\/\/\/\/\/\/\/\/\/\/\/\/\/\/\/\/\/\/\/\/\/\-
%  -X-X-X-X-X-X-X-X-X-X-X-X-X-X-X-X-X-X-X-X-X-X-X-X-X-X-X-X-X-X-X-X-X-X-X-X-X-X-X-X-X-X-X-X-X-X-
% -\/\/\/\/\/\/\/\/\/\/\/\/\/\/\/\/\/\/\/\/\/\/\/\/\/\/\/\/\/\/\/\/\/\/\/\/\/\/\/\/\/\/\/\/\/\/\/\/\/\/\/\/\/\/\/\/\/\/\/\/\/\/\/\/\/\/\/\/\/\/\/\/\/\/\/\/\/\/\/\/\/-
\documentclass[10pt]{article}
\usepackage{geometry} 
\usepackage{indentfirst}
\usepackage{hyperref}
\usepackage{color}
\usepackage{comment}
\usepackage[pdftex]{graphicx}  
\usepackage{caption}
\usepackage{natbib}
\usepackage{mathtools}
\usepackage{units}
\usepackage{booktabs}
\usepackage{authblk}
\renewcommand{\baselinestretch}{1.5}
\geometry{a4paper} 
\bibliographystyle{apalike}


%nouvelle commande pour les commentaires
\newcommand{\sme}[1]{\textcolor{red}{\bf #1}}



%\usepackage[utf8]{inputenc} => pour les accents

% -/\/\/\/\/\/\/\/\/\/\/\/\/\/\/\/\
%\chapter
%\section
%\subsection
%\subsubsection
%\subsubsection
%\paragraph
%\subparagraph
% -\/\/\/\/\/\/\/\/\/\/\/\/\/\/\/\/

%----------------------------------------
%AUTHORS
%----------------------------------------

\title{Heterosis and genetic load in an ExPVP partial diallel cross} 

\author[1]{Jinliang Yang\thanks{jolyang@ucdavis.edu}}
\author[1,2,3]{Sofiane Mezmouk\thanks{smezmouk@ucdavis.edu}}
\author[4]{Rita H. Mumm\thanks{ritamumm@illinois.edu}}
\author[1,2]{Jeffrey Ross-Ibarra\thanks{rossibarra@ucdavis.edu}}
\affil[1]{Department of Plant Sciences, University of California, Davis, CA 95616, USA}
\affil[2]{Center for Population Biology and Genome Center, University of California Davis}
\affil[3]{Current address: KWS SAAT AG, Grimsehlstr. 31, 37555 Einbeck, Germany}
\affil[4]{Department of Crop Sciences, University of Illinois at Urbana-Champaign, Urbana, IL 61801, USA}

\date{}


%-----------------------------------------------------------------------------------------------------------------
%-----------------------------------------------------------------------------------------------------------------
% BEGIN DOCUMENT
%-----------------------------------------------------------------------------------------------------------------
%-----------------------------------------------------------------------------------------------------------------
\begin{document}

\maketitle
%----------------------------------------
% ABSTRACT
%----------------------------------------
\newpage

\section*{Abstract}

%The story is: I have a partial diallel cross with both phenotypic and genotypic data and I observe over dominance at haplotypic level but at the SNP level; everything is explained by dominance. 

Genomic selection (GS) has gained popularity recently as the availability of genome-wide markers has increased. Current methods for GS weigh all the available SNPs equally in model training, without considering their functional differences. Genetic variations detected at evolutionary conserved sites tend to be deleterious and, thus, may be more informative for GS. To utilize this kind of information as a prior into the GS model, we proposed a method to put more weight on evolutionarily constrained sites. As a proof-of-concept, a half diallel population based on 12 diverse inbred lines was used, from which seven phenotypic traits were collected. Some of these traits show high levels of heterosis. After sequencing the 12 founder lines, about 14 million SNPs were discovered and the SNPs were used to identify 502,913 haplotype blocks shared through identity by descent (IBD). A five fold cross-validation experiment was conducted using the summary statistics of the SNP conservation scores, which were computed by evaluating sequences similarity of multiple divergent species, in the IBD blocks as explanatory variables. The results show that the prediction accuracies are significantly better than shuffled data with randomly assigned conservation scores. This study demonstrates the importance of incorporating evolutionary information in GS and its potential use in plant breeding.




%----------------------------------------
% INTRODUCTION
%----------------------------------------
\newpage
\section*{Introduction}
%----------------------------------------

\begin{itemize}
  \item Heterosis and deleterious alleles
  \item Heterosis in maize
  \item Diallel designs and their use for heterosis
  \item What has been done before
  \item What I have
  \item  Main results
\end{itemize}



\textbf{Heterosis was observed long time ago. many theories were proposed to explain it.}


The phenomenon of heterosis or hybrid vigor, refering to the superior performance of hybrids as compared to their parents, was intially documented by Darwin and recovered by Shull and East (Duvick 2001) in 1908. This phenomenon has been observed across taxa, from yeast (Shapira 2014 Heredity) to verterbates (citation). Although heterosis has been broadly applied for food production and has been entensively studied, the underlying mechianisms controlling for it remains unclear. 

Recent studies indicated that the complementation of the deleterious alleles, which fit the dominance model, may play an important role in determining heterosis. Deleterious alleles was arisen from new mutations during meiosis. In maize, about 90 new mutations were generated per meiosis (Clark et al 2005 MBE), majority of which were deemed to be deleterious according to both theoretical (citation) and empirical (citation) evidences. The tendancy of large effect deleterious alleles are determinal or lethal make them subject to be selected against by nature. Therefore, they were mainly maintained in a low frequency (citation, NRG, DF). 

\textbf{Many new mutations, most deleterious}

~90 mutations per meiosis (Clark et al. 2005 MBE, Jiao et al. 2012 Nature Genetics), maize HapMap2 new mutations of large effect (Hufford et al 2012 Nature Genetics, Soletzki 2011 MBE).
Purifying selection retards recovery of diversity in genic regions.


\textbf{Inbreeding depression and segregating deleterious variants and my theories of hybrid breeding}

Maize was domesticated from teonsinte 9000 years ago (citation). The exponential growth since domesication increases number of weakly delterious mutations. Purging: fewer, lower frq premature stop condons in maize and ~8\% FGS > 1 stop condon segregating. The pactice of plant breeding aims to remove these deleterious genetic load and primid the beneficial alleles for agronomical purposes. In order to speed up the breeding process, in practice, diverse accessions of maize germplasm were divided into different heterotic groups (citations) and the gerplasm improvement were conducted in a parrellel fashion. But the down side of the inbreeding is that inbreeding depression will happen. Inbreeding depression occurs when accumulating deleterious alleles, especially in selfing species. Study of the European heterotic groups demonstrated QTLs in the opposite heterotic groups (citation, genetics).    

\begin{itemize}
  \item likely cause of inbreeding depression in Zea
  \item Purging: fewer, lower frq premature stop condons in maize
  \item ~8\% FGS >= 1 stop codon segregating
  \item exponential growth since domestication increases number of weakly deleterious mutations
  \item exponential growth leads to more rare causal variants (Lohmueller 2013), and these explain a larger proportion of Va.
\end{itemize}


\textbf{Theory of residual heterozygosity suggests complementation} 

The practice of removing deleterious alleles in low recombination regions or removing deleterious alleles in tightly linked regions become less effective. For example, residual heterozygosity in RILs correlates negatively with recomination (Gore et al. 2009 Science, McMullen et al. 2009 Science). Low recombination is effective over long time scales (Haddrill et al. 2007 Genome Biology). Deleterious alleles not correlated with recombination (G3). QTL for heterosis enriched in centromeric regions (Lariepe et al, 2012 Genetics).
Selection in high recombination regions improve inbreds?
Haplotype blocks in low recombination maintain heterosis (Duvick 2005, Advances in agronomy)


But, Low Fst and few fixed deleterious variants among groups. 

GWAS hits enriched for genes with deleterious alleles. 1) no enrichment for individual deleterious SNPs (too rare); 2) all traits show enrichment at the gene level; 3) no gene enrichment for synoymous SNPs.











In the past centry, researchers proposed many genetic models, including dominance, overdominance and epistasis to explain it (Birchler et al. 2003; Goff and Zhang 2013). 

Under certain circumstances, some of these genetic models might be favored. For example, an overdominant locus with heterozygous loss-of-function allele in tomato elevated yield by up to 60\% (Krieger et al. 2010). However, in general, heterosis could not be adequately explained by a single gene or a simple model (Birchler et al. 2003).


The fate of deleterious alleles may be impacted by mating system and recombination rate.   


pesudo-overdominance model

lack of recombination in low recombination region. 

The cumulative effect of deleterious mutations 


The complementation of deleterious alleles in different heterotic groups tend to explain it very well.

Previous study observed that an excess deleterious SNPs.

However, because most of the deleterious allele are minor effect. it makes it hard to detect. We therefore, take a genomic enabled prediction approach. 



\textbf{Why we care about deleterious variants. Discuss results of Mezmouk 2014. We are extending this in three ways}  
\begin{itemize}
  \item all deleterious SNPs not just coding
  \item genome-wide not just reduced representation
  \item using GS to test whether they improve prediction
\end{itemize}



previous show deleterious in GWAS set. Here we introduce it to teh genome-wide level.

SIFTMAPP 



%----------------------------------------
% RESULTS
%----------------------------------------
\section*{Results}
%----------------------------------------

\begin{itemize}
  \item Genetic values, heritability, heterosis and combining ability 
  \item SNP calls and annotation; distribution of deleterious mutations along the genome 
  \item correlation between complementation at deleterious SNPs with heterosis and SCA for the different 
  \item IBD region size and general statistics 
  \item IBD correlation with heterosis and SCA
  \item GWAS results at an SNP level and then comparison with haplotypic results 
  \item Analyses of the significant haplotypic bloc to see if there is any pattern
\end{itemize}

\subsection*{IBD region size and general statistics}

Identity by descent was esimated with fasibd.
Average size 44,980 bp (36 to 10,320,000 bp)
Figure S1.



\subsection*{GERP, SIFTMAPP and general statistics}

%GERP statistics
Introduce GERP SNP and GERP elements. For each position of the multiple alignment, the conservation score is computed in rejected substitution by subtracting the estimated evolutionary rate from the neutral rate (citation).
224,087 GERP elements were identified in B73v2. 
29,869,451 bps (1.4\% of the maize genome).


Figure S2. distribution of GERP, GERP vs. MAF.



Figure S3. GERP in gene, exon and intron.


\subsection*{Evolutionary conservation information improved prediction accuracies}

Figure S4. GERP in IBD design with additive and dominance model.




Current methods for GS weigh all the available SNPs equally in model training, without considering their functional differences. Genetic variations detected at evolutionary conserved sites tend to be deleterious and, thus, may be more informative for GS. To utilize this kind of information as a prior into the GS model, we proposed a method to put more weight on evolutionarily constrained sites. As a proof-of-concept, a half diallel population based on 12 diverse inbred lines was used, from which seven phenotypic traits were collected. Some of these traits show high levels of heterosis. After sequencing the 12 founder lines, about 14 million SNPs were discovered and the SNPs were used to identify 502,913 haplotype blocks shared through identity by descent (IBD). A five fold cross-validation experiment was conducted using the summary statistics of the SNP conservation scores, which were computed by evaluating sequences similarity of multiple divergent species, in the IBD blocks as explanatory variables. The results show that the prediction accuracies are significantly better than shuffled data with randomly assigned conservation scores. This study demonstrates the importance of incorporating evolutionary information in GS and its potential use in plant breeding.


%----------------------------------------
% DISCUSSION
%----------------------------------------
\section*{Discussion}
%----------------------------------------


\begin{itemize}
  \item how do results match with heritability and heterosis?
  \item do we support deleterious model of Mezmouk et al.?
\end{itemize}




%----------------------------------------
% MATERIAL & METHODS
%----------------------------------------
\section*{Materials and methods}
%----------------------------------------


\subsection*{Diallel experimental design by Mumm}
%------------------
A broad population was been created, based on a study by Mikel and Dudley (2006), which spans the range of genetic diversity in U.S. contemporary proprietary dent corn germplasm.  Twelve founder inbred corn lines were selected and crossed in a diallel fashion.  Ten are proprietary inbreds that have come off of Plant Variety Protection (PVP) and represent the lineage of key heterotic germplasm pools used in present-day commercial corn hybrids, and two are predominant public inbreds B73 and Mo17.  This diallel facilitated the creation of F1 hybrids that are adapted to the U.S., fairly elite in performance, and commercially relevant (see Table 1).  The set is diverse enough to facilitate a broad sweep of the heterotic sub-groups that comprise U.S. commercial germplasm.  Of the total 66 F1 hybrids (no reciprocals), 32 hybrids represent F-inbred x M-inbred crosses whereas the others represent within-heterotic-group crosses.

Table 1.

The experimental design includes the 66 F1 hybrids, the 12 inbred parents, and 2 current commercial check hybrids grown in an incomplete block design with 3 replications; hybrids and inbreds were grouped separately.  The test was grown at Urbana, IL in 2009, 2010, and 2011.  Plots consisted of 4 rows (17.5 long, each which is ~1/1000th an acre), with all observations taken from the inside 2 rows to minimize effects of shading and maturity differences from adjacent plots.   

Observations were collected on traits pertaining to grain yield, grain yield components including ear attributes, flowering, plant and ear height, lodging, staygreen, barrenness, and seedling vigor.   Plots were hand harvested to facilitate collection of yield component data. 


\subsection*{Plant material and phenotypic data}
%-------------------
Twelve maize inbred lines, including B73, Mo17 and 10 lines with expired U.S. plant variety protection (LH1, LH123HT, LH82, PH207, 4676A, PHG39, PHG47, PHG84, PHJ40, PHZ51) were crossed in a partial diallel fashion covering all one way combinations \sme{(CITE)}. 
 Both inbred lines and the 66 resulting hybrids were field evaluated in Urbana (Illinois, USA) from 2009 to 2011 with three replicates each year, organized in incomplete blocks.\\
Phenotypic data was collected for plant height (PHT, in cm), ear height (EHT, in cm), days to 50\% silking (DTS), days to 50\% pollen shed (DTP), anthesis-silking interval (ASI, in days), grain yield adjusted to 15.5\% moisture (adj GY, in bu/A), and test weight (TWT, in pounds) \sme{(CITE)}.% Question for RHM: Should we site a paper for the phenotypic data? 

\subsection*{Analyses}
%-------------------

\subsubsection*{Inbred line sequencing and SNP calls}

% wet lab
DNA from the twelve inbred lines was CTAB extracted \citep{Doyle1987} and Covaris sheared for Illumina library preparation. The DNA libraries were then sequenced to an average coverage of 10X \sme{(say where the sequencing was done?)}.

%Read mapping
Raw paired reads (reverse and forward for each sequence), were trimmed for adapter contamination with Scythe package (\url{https://github.com/vsbuffalo/scythe}) which calculate the probability of having a contamination given the adapter sequence, the number of mismatches and sequence quality. The reads were then trimmed for quality and sequence length ($\geq 20$ nucleotides) with Sickle package (\url{https://github.com/najoshi/sickle}) .

Read pairs, kept after filtering,  were mapped to the maize B73 reference genome (AGPv2) with bwa-mem \citep{Li2009B}. Reads, with mapping quality (MAPQ) higher than 10 and with a best alignment score higher than the second best one, were kept for further analyses.

%SNP calling
Single nucleotide polymorphisms (SNPs) were called with mpileup from samtools utilities \citep{Li2009}. To deal with paralogy, which is a major problem in maize sequence mapping \citep{Chia2012}, all SNPs were filtered to a) be heterozygote in less than 3 inbred lines, b) have a mean minor allele depth over all genomes of at least 4, c) have a mean depth over all individuals lower than 30 and d) have missing/heterozygote alleles in less than 6 inbred lines (allelic information for at least 15 hybrids in the partial diallel design). 

%SNP results after filtering
After filtering 13,782,809 are kept for further comparisons, including 1,909,416 genic SNPs and 361,280 in protein coding regions). \\
We estimated the allelic error rate by first comparing our genotype calls to those of 41,292 overlapping SNPs on the maize SNP50 bead chip \citep{Heerwaarden2012};  we then compared our SNP alleles for B73 and Mo17 with the 10,426,715 SNP previously identified in HapMap2 \citep{Chia2012}; finally, we compared our SNPs to 180,313 overlapping SNPs identified through Genotyping by sequencing \citep{Romay2013}. The comparisons showed 99.12\% allele similarity with shared SNPs previously identified. 

All missing alleles were then imputed with BEAGLE package \citep{Browning2009} and identity by descent regions (IBD) between the 12 inbred lines were identified with BEAGLE's fastIBD method \citep{Browning2011}. The pairwise IBD region starts and ends were used to delimit haplotipic blocs where several inbred lines shared a homogenous haplotypes.%I need to clarify this last sentence.
\subsubsection*{SNP annotation}

The SNPs were annotated as synonymous and non-synonymous with the software polydNdS from the analysis package of libsequence  \citep{Thornton2003} using the first transcript of each gene in B73 5b filtered gene set. Deleterious effects of amino acid changes were then predicted with both SIFT \citep{Ng2003, Ng2006} and MAPP \citep{Stone2005} software packages as described by \citet{mezmouk2014}.

\sme{Gerp predictions/IBD calling?}

How to call IBD?


\subsubsection*{Phenotypic data analyses}

Best Linear Unbiased Estimation (BLUE) of the genetic effects were calculated with ASReml-R \sme{(CITE)} following the linear model: 
%
\[y_{ijkl} = \mu + \varsigma_{i} + \delta_{ij} + \beta_{jk} + \alpha_{l} +  \varsigma_{i} \cdot \alpha_{l} + \varepsilon\]
%
were 
$y_{ijkl}$ is the phenotypic value of the $l^{th}$ genotype evaluated in the $k^{th}$ block of the $j^{th}$ replicate within the $i^{th}$ environment; 
$\mu$, the whole mean; 
$\varsigma_{i}$, the fixed effect of the $i^{th}$ environment;
$\delta_{ij}$, the fixed effect of the $j^{th}$ replicate nested in the $i^{th}$ environment; 
$\beta_{jk}$, the random effect of the $k^{th}$ block nested in the $j^{th}$ block; 
$\alpha_{l}$, the the fixed genetic effect  of the $l^{th}$ individual; 
$\varsigma_{i} \cdot \alpha_{l}$, the interaction effect of the $l^{th}$ individual with the $i^{th}$ environment; 
$\varepsilon$, the model residuals.

All fixed effects, including the genotype onces, were significant (Wald test p-value<0.05) for all traits except ASI for which the effect of the replicates within environments were not significant.

Heterosi for each hybrid was then estimated by both best- and mid-parent heterosis ($BPH$ and $MPH$, respectively):
%
\[ MPH_{ij}=\hat{G_{ij}}-\frac{1}{2}(\hat{G_{i}}+\hat{G_{j}}) \]
\[ BPH_{min,ij}=\hat{G_{ij}}-min(\hat{G_{i}} ,\hat{G_{j}}) \] 
\[ BPH_{max,ij}=\hat{G_{ij}}-max(\hat{G_{i}} ,\hat{G_{j}}) \]
%
where $\hat{G_{ij}}$, $\hat{G_{i}}$ and $\hat{G_{j}}$ are the genetic values of the hybrid and its two parents $i$ and $j$. $BPH_{min}$ was used instead of $BPH_{max}$ for days to anthesis.\\

Finally, general and specific combining ability (GCA ans SCA) respectively were estimated following \citet{Falconer1996}.

\subsubsection*{Association mapping}
SNP association with heterosis (BPH and MPH) was tested assuming dominance/recessivity of the reference allele or assuming overdominance where only the heterozygote alleles are expected to be significant. For each SNP, root mean square error were used to select the best fitting model. 
Haplotype association with heterosis were tested comparing the heterozygote alleles to all homozygote ones all confounded. 


\subsubsection*{Genomic-enabled prediction with GERP scores in IBD regions}

In this study, we used summary statistics of GERP scores in the IBD regions to fit the genomic selection model for phenotypic prediction. The sum of the conservation score at the detected variation loci in a given IBD block was used as the explainatory variable. A customized python script was used to compute the conservation statistics of each IBD block across the diallel population.

To rule out the prossibility that SNP in genic regions are more conserved than intergenic SNPs. We elected the SNPs in genic regions and did the circular shuffling to random assign GERP scores the the same set of the selected SNPs.




Four summary statistics were used for each IBD block, they are 1) number of conserved SNPs in additive mode, 2) sum of the conservation scores in additive mode, 3) number of conserved SNPs in dominant mode and 4) sum of the conservation scores (GERP) in dominant mode.
First, a python script gerpIBD.py was developed to compute the conservation statistics of each IBD block across a diallel population. The scores were normalized across samples. The script was uploaded to the repo of zmSNPtools. Use -help for more detail of the program.
Second, gerp scores were circular shuffled and were used to gen- erate N set of random assigned conservation statistics for each IBD block.
Third, seven phenotypic traits were trained with the conserva- tion statistics in IBD block as genotype. The results were compared with circular shuffled genotypes. The results showed the real data fit always better than the shuffled genotypes.
Finally, we trained the GS model with subset of the phenotypic data and predict the validation set using a 5-fold cross-validation strategy.

\subsubsection*{Stastistical model for Genomic-enabled prediction}

%----------------------------------------
% REFERENCES
%----------------------------------------
\clearpage
\bibliography{Diallel}

\end{document}
%-----------------------------------------------------------------------------------------------------------------
%-----------------------------------------------------------------------------------------------------------------
% END DOCUMENT
%-----------------------------------------------------------------------------------------------------------------
%-----------------------------------------------------------------------------------------------------------------
% -/\/\/\/\/\/\/\/\/\/\/\/\/\/\/\/\/\/\/\/\/\/\/\/\/\/\/\/\/\/\/\/\/\/\/\/\/\/\/\/\/\/\/\/\/\/\/\/\/\/\/\/\/\/\/\/\/\/\/\/\/\/\/\/\/\/\/\/\/\/\/\/\/\/\/\/\/\/\/\/\/\-
%  -X-X-X-X-X-X-X-X-X-X-X-X-X-X-X-X-X-X-X-X-X-X-X-X-X-X-X-X-X-X-X-X-X-X-X-X-X-X-X-X-X-X-X-X-X-X-
% -\/\/\/\/\/\/\/\/\/\/\/\/\/\/\/\/\/\/\/\/\/\/\/\/\/\/\/\/\/\/\/\/\/\/\/\/\/\/\/\/\/\/\/\/\/\/\/\/\/\/\/\/\/\/\/\/\/\/\/\/\/\/\/\/\/\/\/\/\/\/\/\/\/\/\/\/\/\/\/\/\/-